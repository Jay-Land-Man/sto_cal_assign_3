\documentclass[answers]{exam}
\usepackage[margin=0.75in]{geometry}
\usepackage[parfill]{parskip}
\usepackage{amsmath}
\usepackage{amssymb}
\usepackage{amsthm}
\usepackage{amsfonts}
\usepackage{mathrsfs}
\usepackage[utf8]{inputenc}

\title{Assignment 3}
\date{April 2019}

\begin{document}
	

\section*{Exercise Sheet 6}

\subsection*{Question 5}

\begin{parts}
	
	\part
	
	Show that $X$ is not attainable.
	
	\begin{solution}
		
		If X is attainable, then  there exist $V_t(\theta)$ such that $V_t(\theta)=  X_t$. Hence $\bar{V}_t(\theta)=  \bar{X}_t$ 
		
		$$\mathbb{Q}(\omega_{k}) = \mathbb{E}_\mathbb{Q}[\bar{X}]= \mathbb{E}_\mathbb{Q}[\bar{V}_T(\theta)]= \bar{V}_0(\theta)$$
		
		$$\mathbb{Q'}(\omega_{k}) = \mathbb{E}_\mathbb{Q'}[\bar{X}]= \mathbb{E}_\mathbb{Q'}[\bar{V}_T(\theta)]= \bar{V}_0(\theta)$$
		
		but
		
		$$\mathbb{Q}(\omega_{k}) \neq \mathbb{Q'}(\omega_{k})$$
		
		Therefore under no abitrage, there is no portfolio $V(\theta)$ which attains X 
		
	\end{solution}
	
	\part
	\renewcommand{\labelenumi}{\roman{enumi}}
	\begin{enumerate}
		\item 
		
		Show that there is a vector $\xi = (\xi_1,...,\xi_K)$ with property that $\xi.\bar{V}_T(\theta) = 0$ for each trading strategy , but has $\xi.X > 0$
		
		\begin{solution}
			
			Define $L =  \{\bar{V}_T(\theta):\text{all } \theta\}$ and $K = \{\bar{X}\}$ then $K\cap L =\emptyset$ by (a). By the hyperplane separation theorem, there exists $\xi \in \mathbb{R}^n$ such that.  
			$$\xi.\bar{V}_T(\theta) = 0$$
			$$\xi.\bar{X} > 0$$
			hence $\xi.X > 0$ by definition of $X$
			
		\end{solution}
		
		\item
		
		Show that $\mathbb{Q(\omega_k)} = q'_k $ defines a probability measure $\Omega$ which is equivalent to $\mathbb{P}$
		
		\begin{solution}
			
			\begin{align*}	
			q'.\bar{V}_T(\theta) &= (q + \lambda\xi)\bar{V}_T(\theta) \text{   (but } \xi. \bar{V}_T(\theta) = 0) \\
			q'.\bar{V}_T(\theta) &= q.\bar{V}_T(\theta)	
			\end{align*}	
			
			If $q'_k = 1, \text{then } q'_j = 0 \text{ for all } j \neq k, \text{take } \bar{V}_T(\theta)=1$ for all $\omega$. Then $q_k = 1 \text{ and } q_j= 0 \text{ for all } j \neq k $. Thus $\mathbb{Q'} \text{ is equivalent to } \mathbb{Q} \text{ and hence it is also equivalent to } \mathbb{P} . $
			
			
		\end{solution}
		
		\item 
		Show that $\mathbb{Q'}$ is in fact a risk-neutral measure.
		
		\begin{solution}
			
			\begin{align*}
			\mathbb{E}_\mathbb{Q'}[\bar{V}_T(\theta)]
			&=\sum_{k=1}^{K}\mathbb{Q'}(\omega_{k}) \bar{V}_T(\omega_k)\\
			&=\sum_{k=1}^{K}q'_k\bar{V}_T(\omega_k)\\
			&=\sum_{k=1}^{K}q_k\bar{V}_T(\omega_k)+ \lambda \sum_{k=1}^{K} \xi_k \bar{V}_T(\omega_k)  \\
			&=\sum_{k=1}^{K}\mathbb{Q}(\omega_{k}) \bar{V}_T(\omega_k) + \lambda \sum_{k=1}^{K} \xi_k \bar{V}_T(\omega_k)\\
			&=\mathbb{E}_\mathbb{Q}[\bar{V}_T(\theta)] + \lambda \xi. \bar{V}_T(\theta)
			\text{   (but } \xi. \bar{V}_T(\theta) = 0)\\
			&=\mathbb{E}_\mathbb{Q}[\bar{V}_T(\theta)]\\
			&=\bar{V}_0(\theta)
			\end{align*}
			
			Thus $\mathbb{Q'}$ is in fact a risk-neutral measure.
			
		\end{solution}
		
		\item 
		Show that $\mathbb{E}_\mathbb{Q'}[\bar{X}] \neq \mathbb{E}_\mathbb{Q}[\bar{X}]$
		
		
		\begin{solution}
			
			\begin{align*}
			\mathbb{E}_\mathbb{Q'}[\bar{X}]
			&=\sum_{k=1}^{K}\mathbb{Q'}(\omega_{k})[\bar{X}]\\
			&=\sum_{k=1}^{K}q_k\bar{X} + \lambda \sum_{k=1}^{K} \xi_k \bar{X}  \\
			&=\sum_{k=1}^{K}\mathbb{Q}(\omega_{k})\bar{X} + \lambda \sum_{k=1}^{K} \xi_k \bar{X}  \\
			&=\mathbb{E}_\mathbb{Q}[\bar{X}] + \lambda \xi. \bar{X} \text{ (but } \xi. \bar{X} >0 )\\
			&\neq \mathbb{E}_\mathbb{Q}[\bar{X}] 
			\end{align*}
			
		\end{solution}
		
	\end{enumerate}
\end{parts}


\end{document}