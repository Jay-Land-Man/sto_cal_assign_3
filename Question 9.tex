\documentclass[12pt, letterpaper]{article}
\usepackage[margin=1in]{geometry}
\usepackage{enumitem} 
\usepackage{amsmath} 
\usepackage{amssymb} 
\usepackage{amsthm} 
\usepackage{mdframed}
\usepackage{parskip}

\newcommand\myeq{\mathrel{\overset{\makebox[0pt]{\mbox{\normalfont\tiny\sffamily DCT}}}{=}}}

 
\title{Stochastic Calculus in Finance\\
		\large Exercise Sheet 5} 
\author{Jayson Landman}
\begin{document}
	\section*{Question 9}
	 
	 \begin{enumerate} [label = \textbf{\alph*)}]
		\item If $\frac{d\nu}{d\mu}$ and $\frac{d\eta}{d\nu}$ exist then $\eta \ll \nu \ll \mu$. For any $A \subseteq \Omega$ we have:
		\begin{align*}
			\mu(I_A fg) & = \int_A fg \; d\mu \\
			& = \int_A \frac{d\nu}{d\mu} g \; d\mu \tag{$f = \frac{d\nu}{d\mu}$}\\
			& = \int_A g \; d\nu \\
			& = \int_A \frac{d\eta}{d\nu} \; d\nu \tag{$g = \frac{d\eta}{d\nu}$} \\
			& = \int_A 1 \; d\eta \\
			& = \eta(I_A)
		\end{align*}
		But notice that $\eta \ll \mu$. So we also have 
		\begin{align*}
			\mu(I_A f') & = \int_A f' \; d\mu \\
			& = \int_A \frac{d\eta}{d\mu}  \; d\mu \tag{$f' = \frac{d\eta}{d\mu}$}\\
			& = \int_A 1 \; d\eta \\
			& = \eta(I_A)
		\end{align*}
		$\mu(I_A\frac{d\eta}{d\nu}\frac{d\nu}{d\mu}) = \mu(I_A\frac{d\eta}{d\mu})$	$\mu$-almost everywhere. 
		\begin{align*}
			\frac{d\eta}{d\mu} & = \frac{d\eta}{d\nu}\frac{d\nu}{d\mu} \tag{$\mu$-almost everywhere}  
		\end{align*}
		\textbf{Need to check with prof that this is an almost everywhere result}
		\newpage
		\item Existence of $\frac{d\nu}{d\mu} \rightarrow \nu \ll \mu$. Therefore $\mu(A) = 0 \rightarrow \nu(A) = 0.$ \newline
		If $\frac{d\nu}{d\mu} > 0 \; \; \mu$-a.e. then $\nu(A) = 0 \rightarrow \mu(A) = 0$ and $\mu \ll \nu$. \newline $\nu$ and $\mu$ are equivalent measures.
		
		For any $A \subseteq \Omega$
		\begin{align*}
			\mu(I_A fg) & = \int_A fg \; d\mu \\
			& = \int_A \frac{d\nu}{d\mu} g \; d\mu \tag{$f = \frac{d\nu}{d\mu}$}\\
			& = \int_A g \; d\nu \\
			& = \int_A \frac{d\mu}{d\nu} \; d\nu \tag{$g = \frac{d\mu}{d\nu}$} \\
			& = \int_A 1 \; d\mu \\
			& = \mu(I_A)
		\end{align*}
		
		$\mu(I_A) = \mu(\frac{d\mu}{d\nu}\frac{d\nu}{d\mu}I_A)$	$\mu$-almost everywhere. \newline This implies that $\frac{d\mu}{d\nu}\frac{d\nu}{d\mu}$ is the same as the ghost function of $1$ under $\mu$. 
		\begin{align*}
			\frac{d\mu}{d\nu}\frac{d\nu}{d\mu} & = 1 \tag{$\mu$-almost everywhere} \\		
			\frac{d\mu}{d\nu}\frac{d\nu}{d\mu}(\frac{d\nu}{d\mu})^{-1} & = (\frac{d\nu}{d\mu})^{-1}  \tag{$\mu$-almost everywhere} \\					
				\frac{d\mu}{d\nu} & = \frac{1}{\frac{d\nu}{d\mu}} \tag{$\mu$-almost everywhere} 
		\end{align*}
		
	\end{enumerate}

\end{document}