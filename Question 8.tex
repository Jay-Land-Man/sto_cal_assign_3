\documentclass[12pt, letterpaper]{article}
\usepackage[margin=1in]{geometry}
\usepackage{enumitem} 
\usepackage{amsmath} 
\usepackage{amssymb} 
\usepackage{amsthm} 
\usepackage{mdframed}
\usepackage{parskip}

\newcommand\myeq{\mathrel{\overset{\makebox[0pt]{\mbox{\normalfont\tiny\sffamily DCT}}}{=}}}

 
\title{Stochastic Calculus in Finance\\
		\large Exercise Sheet 5} 
\author{Jayson Landman}
\begin{document}
	\section*{Question 8}
	 
	 Suppose $f(X_n) \not \xrightarrow{P} f(X)$. \newline Choose $\langle f(X_{n_k})\rangle_k$ such that $\mathbb{E}[|f(X_{n_k}) - f(X)|\wedge 1] > \epsilon \: \forall k$. \newline
	If $\langle f(X_{n_{j_k}}) \rangle_j$ is a subsubsequence we cannot have $f(X_{n_{k_j}}) \rightarrow f(X)$ almost surely. \newline
	Otherwise $|f(X_{n_{k_j}}) - f(X)| \wedge 1 \rightarrow 0$ and by DCT $\mathbb{E}[|f(X_{n_{k_j}}) - f(X)| \wedge 1] \rightarrow 0$. \newline
	The contrapositive result implies $f(X_n) \xrightarrow{P} f(X)$ if every subsequence $\langle f(X_{n_k}) \rangle_k$ of $\langle f(X_n) \rangle$ has a subsubsequence $\langle f(X_{n_{k_j}}) \rangle_j$ that converges to $f(X)$ almost surely.
	 
	We are given that subsequences $\langle X_{n_k} \rangle_k$ of $\langle X_n \rangle$ have subsubsequences $\langle X_{n_{k_j}} \rangle_j$ that converge to $X$ almost surely. Similarly, $f(X_{n_{k_j}}) \rightarrow f(X)$ because $f$ is almost surely continuous at $X$.

\end{document}