\documentclass[answers]{exam}
\usepackage[margin=0.75in]{geometry}
\usepackage[parfill]{parskip}
\usepackage{amsmath}
\usepackage{amssymb}
\usepackage{amsthm}
\usepackage{amsfonts}
\usepackage{mathrsfs}
\usepackage[utf8]{inputenc}

\title{Assignment 3}
\date{April 2019}

\begin{document}
	

\section*{Exercise Sheet 6}

\subsection*{Question 7b}

Show that if $A \in \mathcal{F}_\sigma$ then $A \cap \{\sigma \leq \tau\} \in \mathcal{F}_{\sigma \land \tau}$

\begin{solution}
	
	\begin{align*}
	A \cap \{\sigma \leq \tau \} \cap \{\sigma \land \tau \leq t \} 
	&= A \cap \{\sigma \leq \tau \} \cap (\{t \leq \tau \} \cup \{\tau < t \} )\\
	&= A \cap \{ \sigma \leq t\} \cap (\{ \sigma \leq t \leq \tau\} \cup (\{ \sigma < \tau < t\}) \\
	&=A \cap \{ \sigma \leq t\} \cap \{ \sigma \land t \leq \tau \land t \}
	\end{align*}
	
	Since $\{ \sigma \leq t\}$ is $ \in \mathcal{F}_t$ and $\{ \sigma \land t \leq \tau \land t \}$ is $ \in \mathcal{F}_t$. Then $\{ \sigma \leq t\} \cap \{ \sigma \land t \leq \tau \land t \} = \{\sigma \leq \tau \} \cap \{\sigma \land \tau \leq t \}$ is $ \in \mathcal{F}_t  $. Hence $\{\sigma \leq \tau \},\{\sigma = \tau \} \in \mathcal{F}_{\sigma \land \tau}$ 
	
\end{solution}
 


\end{document}