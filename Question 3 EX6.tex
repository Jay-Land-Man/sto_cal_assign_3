\documentclass[answers]{exam}
\usepackage[margin=0.75in]{geometry}
\usepackage[parfill]{parskip}
\usepackage{amsmath}
\usepackage{amssymb}
\usepackage{amsthm}
\usepackage{amsfonts}
\usepackage{mathrsfs}
\usepackage[utf8]{inputenc}

\title{Assignment 3}
\date{April 2019}

\begin{document}
	

\section*{Exercise Sheet 6}

\subsection*{Question 3}
\begin{parts}
	
	\part
	
	Determine the risk-free interest rate, given that there is no arbitrage.
	
	\begin{solution}
		
		$$\begin{pmatrix} 
		1 \\
		2.9\\
		1.1  
		\end{pmatrix} = \frac{1}{1+r} \begin{pmatrix} 
		1+r & 1+r \\
		3 & 4 \\
		2 & 1
		\end{pmatrix} \begin{pmatrix} 
		q_1 \\
		q_2 
		\end{pmatrix}$$
		
		Then
		
		\begin{align*}
		1 &= q_1 +q_2\\
		2.9 &= \frac{3}{1+r}q_1 +\frac{4}{1+r}q_2\\
		1.1 &= \frac{2}{1+r}q_1 +\frac{1}{1+r}q_2
		\end{align*}
		
		Hence 
		
		\begin{align*}
		r &=\frac{1}{4}\\
		q_1 &= \frac{3}{8}\\
		q_2 &= \frac{5}{8}
		\end{align*} 
		
	\end{solution}
	
	\part
	
	Is the market complete?
	
	\begin{solution}
		
		Yes, there exist a unique equivalent martingale measure.	
		
	\end{solution}
	
	\part
	
	A contingent claim pays out 2.00 in state $\omega_1$ and 3.00 in state $\omega_2$. What is the $t = 0$ value of this claim?
	
	\begin{solution}
		
		The value of the claim is 
		
		$$\frac{2}{1+r}q_1 + \frac{3}{1+r}q_2 = \frac{21}{10}$$
		
	\end{solution}
	
	
\end{parts} 


\end{document}